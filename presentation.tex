\documentclass[compress,handout,10pt]{beamer}

\newlength{\wideitemsep}
\setlength{\wideitemsep}{\itemsep}
\addtolength{\wideitemsep}{100pt}
\let\olditem\item
\renewcommand{\item}{\setlength{\itemsep}{0.5\baselineskip}\olditem}

\usetheme{AnnArbor}
\usecolortheme{crane}
\usefonttheme[onlymath]{serif}

\usepackage{float}
\floatstyle{boxed}
\usepackage{colortbl}
\usepackage{mathpazo}
\usepackage{graphicx}
\usepackage{movie15}
\usepackage{bm}
\usepackage{verbatim}
\usepackage{comment}
\usepackage{caption}
\usepackage{subcaption}
\captionsetup[subfigure]{labelformat=empty}
\captionsetup[figure]{labelformat=empty}

\newcommand{\mygreen}{\color{green!50!black}}
\newcommand{\myblue}{\color{blue}}
\newcommand{\myred}{\color{red}}
\newcommand{\mycolor}{\color{red}{c}\color{blue}{o}\color{green}{l}\color{orange}{o}\color{cyan}{r}}
\newcommand{\mysize}{\scriptsize{s}\small{i}\normalsize{z}\Large{e}}
\newcommand{\myshape}{\textcircled{s}\textit{h}\texttt{a}\textsf{p}\textsc{e}}

\xdefinecolor{titlecolor}{rgb}{.855,.647,.125}
\setbeamercolor{frametitle}{fg=titlecolor}
\setbeamerfont{frametitle}{series=\bfseries}
\setbeamercolor{normal text in math text}{parent=math text}

\setbeamertemplate{navigation symbols}{} %gets rid of navigation symbols
\setbeamertemplate{footline}[frame number]
\beamertemplateshadingbackground{blue!5}{yellow!10}

\title{{\color{blue} \LARGE Logistics of Constructing a Washington D.C.-Baltimore Rapid Transit Line\newline} }

\subtitle{{\large Washington Metropolitan Area Transit Authority \newline \large Maryland Transit Administration} }

\author{ 
%    \vspace{5pt}
    {\bf{Participant:}} \\ 
Danni Tang \\ 
    \vspace{5pt}
} 
\institute{Johs Hopkins University}

\date{\mygreen Last Complied on \today} 

\begin{document}

\begin{frame}[plain]
    \titlepage
\end{frame}

\begin{frame}
    \frametitle{Table of Contents}
    \tableofcontents
\end{frame}

\section{Introduction}

\begin{frame}
    \frametitle{Sponsors}
    Washington Metropolitan Area Transit Authority (WMATA):
    \vspace{7pt}
             \begin{enumerate}
                 \item transportation agency created by the Library of Congress
		 \item operates in the District of Columbia, Maryland, and the Commonwealth of Virginia
                 \item rapid transit service (Metrorail)
                 \item bus services (Metrobus)
                 \item paratransit (MetroAccess)
                 \item currently contructing new lines in Virginia (Silver Line) and in Maryland surburbs of D.C. (Purple Line)
             \end{enumerate}
\end{frame}

\begin{frame}
    \frametitle{Sponsors (cont.)}
    Maryland Transit System (MTA Maryland):
    \vspace{7pt}
             \begin{enumerate}
                 \item transportation agency operated by the state of Maryland
                 \item operates in the Baltimore-Washington Metropolitan area
                 \item numerous bus lines
                 \item Light Rail
                 \item Metro Subway
		 \item MARC train
             \end{enumerate}
\end{frame}

\begin{frame}
    \frametitle{Relevance}
    Problem area:
    \vspace{7pt}
     \begin{itemize}
	\item D.C. and Baltimore have similar worker populations
\begin{table}[ht]
\caption{Workers Who Use Public Transportation}   % title of Table
\centering % used for centering table
\begin{tabular}{c c c} % centered columns (3 columns)
\hline\hline %inserts double horizontal lines
City & \# of workers & \# of cars, trucks, or vans \\ [0.5ex]
\hline  % inserts single horizontal line
Washington, D.C. & 293,532 & 127,494 \\% inserting body of the table
Baltimore & 269,917 & 186,961 \\ [1ex] % [1ex] adds vertical space
\hline %inserts single line
\end{tabular}
\label{table:workers} % is used to refer this table in the text
\end{table}
	\item 43\% of D.C. workers commute in cars, trucks, or vans
	\item 69\% of Baltimore workers commute in cars, trucks, or vans
	\item it is apparent that large populations of workers of both cities rely on vehicles to commute
	\item a subway line between the two cities would greatly reduce traffic volume, jams, and accidents
	\item sponsors would find this model relevant
     \end{itemize}
\end{frame}

\begin{frame}
    \frametitle{Problem Statement}
    \begin{itemize}
        \item WMATA has no plans to expand the Metrorail system to the city and suburbs surrounding Baltimore
	\item MTA Maryland's Metro Subway system only operates within city limits
	\item residents of Greater Washington-Baltimore Metropolitan area have limited access to public transportation to travel between the two cities
	\item current public transportation methods:
	\begin{itemize}
		\item AMTRAK fares too expensive for daily commute
		\item MARC operates rush hours on weekdays
	\end{itemize}
	\item both sponsors operate under two separate government agencies
	\item our task is to provide a model that can predict the operating capacity for a such a line based on published transportation statistics
    \end{itemize}
\end{frame}

\begin{frame}
    \frametitle{Deliverables: From Sponsor to Team}
    \begin{enumerate}
        \item most recent data and statistics from Maryland Department of Transportation by Oct 19, 2012
	\begin{itemize}
		\item contingency plan: if data not received by the assigned time, we will obtain data published on the Department of Transportation website
	\end{itemize}
	\item computing resources
	\item timely responses to inquiries
	\item small expenses relevant to work
    \end{enumerate}
\end{frame}

\begin{frame}
    \frametitle{Deliverables: From Team to Sponsor}
    \begin{enumerate}
        \item mathematical model of traffic flow at various hours of the day (morning, noon, evening)
	\item traffic flow will model highways I-495 and I-95
	\item analytical report on the results of traffic flow model to determine if a subway line is viable
	\item time permitting, design of the subway line
	\item R package with documentations and codes to reproduce test results
	\item technical report and presentation summarizing the work done
    \end{enumerate}
\end{frame}

\section{Principles}
\begin{frame}
    \frametitle{Seven Basic Principles}
     \begin{enumerate}
         \item Set the context 
         \item Choose effective examples and analogies
         \item Choose vocabulary to suit your readers
         \item Decide whether to present \#s in text, tables, or figures
         \item Report and interpret \#s in the text
         \item Specify the direction \emph{and} size of an association between variables
         \item For many \#s, summarize overall pattern 
     \end{enumerate}
\end{frame}

\section{Tools}
\begin{frame}
    \frametitle{Creating Effective Tables}
\end{frame}

\section{Arguments from Scale}

\begin{frame}
    \frametitle{Example: Cost of Packaging}
\end{frame}

\section{Graphical Methods}
\begin{frame}
    \frametitle{Example: The Nuclear Mission Arms Race}
\end{frame}

\section{Basic Optimization}
\begin{frame}
    \frametitle{Example: Maintaining Inventory}
\end{frame}

\begin{frame}
\bibliographystyle{plain}
%%\renewcommand\bibname{Selected Bibliography Including Cited Works}
\nocite{*}
\bibliography{biblio}{}
\end{frame}
\end{document}
